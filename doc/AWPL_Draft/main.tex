\documentclass[10pt, twocolumn]{IEEEtran}

% Packages
\usepackage{bm}
\usepackage{url}
\usepackage{bbm}
\usepackage{cite}
\usepackage{array}
\usepackage{ifthen}
\usepackage{xspace}
\usepackage{dsfont}
\usepackage{siunitx}
\usepackage{balance}
\usepackage{amsmath}
\usepackage{amssymb}
\usepackage{multicol}
\usepackage{amsfonts}
\usepackage{mathrsfs}
\usepackage{booktabs}
\usepackage{graphicx}
\usepackage{caption}
\usepackage{setspace}
\usepackage{hyperref}
\usepackage{makecell}
\usepackage{footnote}
\usepackage{verbatim}
\usepackage{algorithm}
\usepackage{subcaption}
\usepackage{glossaries}
\usepackage{soul,xcolor}
\usepackage[T1]{fontenc}
\usepackage{algpseudocode}
\usepackage{algcompatible}
\usepackage[normalem]{ulem}
\usepackage{multirow,enumitem}

% Setting up packages
\captionsetup{font=footnotesize}
\setlength{\textfloatsep}{1.5pt}
\sisetup{detect-all,range-phrase=--,range-units=single,group-separator={,}}

% Initializing new commands
\newcommand{\sst}[1]{\st{#1}}
\newcommand{\tot}{\mathrm{tot}}
\newcommand{\tfrm}{T_{\mathrm{fr}}}
\newcommand{\beam}[1]{\mathcal B_{#1}}
\newcommand{\add}[1]{\textcolor{red}{#1}}
\newcommand{\size}[1]{\left | #1 \right|}
\newcommand{\bk}[1]{\textcolor{blue}{[BK: #1]}}
\newcommand{\yz}[1]{\textcolor{blue}{[YZ: #1]}}
\newcommand{\ca}[1]{\textcolor{magenta}{[CA: #1]}}
\newcommand{\nm}[1]{\textcolor{magenta}{[NM: #1]}}
\newcommand{\jvk}[1]{\textcolor{magenta}{[JVK: #1]}}
\newcommand{\djl}[1]{\textcolor{magenta}{[DJL: #1]}}
\newcommand{\beambs}[1]{\mathcal B_{{\mathrm t},#1}}
\newcommand{\beamue}[1]{\mathcal B_{{\mathrm r},#1}}
\newcommand{\diag}[1]{\mathrm{diag}\left(#1 \right)}
\newcommand{\numberthis}{\addtocounter{equation}{1}\tag{\theequation}}
\newcommand\mst[2][red]{\setbox0=\hbox{$#2$}\rlap{\raisebox{.45\ht0}{\textcolor{#1}{\rule{\wd0}{2pt}}}}#2}

% Redefining commands
\renewcommand\theadalign{c}
\renewcommand{\tabcolsep}{2pt}
\renewcommand\theadfont{\bfseries}
\renewcommand\cellgape{\Gape[2pt]}
\renewcommand\theadgape{\Gape[2pt]}

% Title section
\title{Multipath Clustering Evaluations of \SI{28}{\giga\hertz} V2X Propagation via 3D Ray-Tracing Simulations}
\author{Bharath Keshavamurthy\IEEEauthorrefmark{1}, Yaguang Zhang\IEEEauthorrefmark{2}, Christopher R. Anderson\IEEEauthorrefmark{3},\\Nicol\`{o} Michelusi\IEEEauthorrefmark{1}, David J. Love\IEEEauthorrefmark{2}, and James V. Krogmeier\IEEEauthorrefmark{2}
\thanks{\IEEEauthorrefmark{1}Electrical, Computer and Energy Engineering, Arizona State University.}
\thanks{\IEEEauthorrefmark{2}Electrical and Computer Engineering, Purdue University.}
\thanks{\IEEEauthorrefmark{3}Electrical Engineering, United States Naval Academy.}
\thanks{A preliminary version of this research was accepted at IEEE ICC $2023$~\cite{ICC}.}
\thanks{Research funded by NSF under grants CNS-1642982 and CNS-2129615.}
\thanks{Source code available on \href{https://github.com/bharathkeshavamurthy/SPAVE-28G.git}{GitHub}\cite{SPAVE-28G-Software}. Dataset available on \href{https://doi.org/10.5281/zenodo.7178597}{Zenodo}\cite{SPAVE-28G-Dataset}.}
\vspace{-12mm}
}

% Content begins
\begin{document}
\bstctlcite{IEEEexample:BSTcontrol}

\maketitle
\thispagestyle{empty}
\pagestyle{empty}
\setulcolor{red}
\setul{red}{2pt}
\setstcolor{red}
\vspace{-12mm}

% Abstract
\begin{abstract}
This letter discusses the propagation characteristics of \SI{28}{\giga\hertz} signals in vehicular environments from the perspective of the datasets collected during a beam-steered measurement campaign on the NSF POWDER testbed as well as ray-tracing simulations on Wireless InSite. In particular, this paper presents analyses of the geometry-induced losses, the multipath clustering attributes, and the Doppler shift behavior of \SI{28}{\giga\hertz} signals in V$\mathbf{2}$X ecosystems resulting from such site-specific propagation modeling, wherein these evaluations are validated and rendered physically-interpretable by $\mathbf{3}$D ray-tracing simulations involving the terrain, structural, and foliage models of the NSF POWDER site in Salt Lake City, UT. Herein, diffuse scattering models, temporal \& spatial dispersion features, and multipath intra- and inter-cluster characteristics are investigated at multiple positions along $\mathbf{6}$ different routes spanning diverse radio environments.
\end{abstract}

% Index terms
\begin{IEEEkeywords}
    Multipath clustering, V$\mathbf{2}$X, Ray-Tracing
\end{IEEEkeywords}
\vspace{-3mm}

% Introduction, Literature survey, and Novelties
\section{Introduction}\label{S1}
Enterprises across sectors have stepped-up their adoption of next-generation radio access technologies to improve yields, automate logistics, optimize supply chains, and streamline operations. The ultra-reliable low-latency guarantees provided by $5$G$+$ and $6$G networks hinges on the efficient utilization of the millimeter wave bands (mmWave: \SIrange{30}{300}{\giga\hertz}), which by virtue of their increased bandwidth relative to the popular mid-band spectrum (C-band: \SIrange{4}{8}{\giga\hertz}) provide significant enhancements in network performance vis-\`{a}-vis data rates and latencies ~\cite{mmWaveSurvey}. However, mmWave signals suffer from increased atmospheric attenuation due to their relatively high free-space pathloss; considerable slow-fading consequences due to obstacles in the signal path; exacerbated multipath fading due to diffuse reflections off surfaces, and diffractions by foliage and building-edges; and lastly, Doppler shift and fast-fading effects, particularly evident in V$2$X settings~\cite{Rappaport}.

% Add literature survey and contributions

The rest of this letter is structured as: Sec.~\ref{S2} describes our propagation modeling activities on the NSF POWDER testbed; Sec.~\ref{S3} details the setup of our ray-tracing evaluations on Wireless InSite; Sec.~\ref{S4} describes our numerical evaluations and the insights gained from pathloss and spatial consistency studies, while Sec.~\ref{S5} discusses our multipath clustering investigations; and finally, Sec.~\ref{S6} outlines our concluding remarks.
\vspace{-3mm}

% Measurement campaign description and Data post-processing procedural explanation
\section{NSF POWDER: Measurements \& Processing}\label{S2}
In this section, we discuss the operations involved in our \SI{28}{\giga\hertz} V$2$X measurement campaign on the NSF POWDER testbed. First, we describe the system calibration process; next, we outline the onsite system deployment procedure; and finally, we detail the post-processing steps involved in setting up the power-delay profiles recorded at the Rx for pathloss evaluations, multipath component extraction and parameter estimation via SAGE, and spatial consistency analyses.

\noindent{\textbf{Pre-deployment Calibration}}: After the Tx and Rx circuits for the sliding correlator channel sounder have been implemented as discussed in \cite{ICC}, a calibration procedure is carried out onsite to map the power calculated from the power-delay profiles recorded by the USRP to reference measured power levels. Calibrating the measurement system before deployment ensures accurate Rx power calculations in the presence of noisy/imperfect circuit components such as the Commscope LDF$4$-$50$A \SI{0.5}{{"}} coaxial cables employed at the Tx which exhibit losses of up to \SI{0.12}{\deci\bel\per\meter} at \SI{2.5}{\giga\hertz}. Under \SI{0}{\deci\bel} and \SI{76}{\deci\bel} USRP gains, using a Keysight variable attenuator, the recorded power-delay profiles are processed to determine the calculated power values mapped to the reference power levels: the results of this procedure are studied in Sec.~\ref{S4}.

\noindent{\textbf{NSF POWDER Deployment}}: The measurement assembly constitutes an autonomous beam-steering controller replicated at both the Tx and the Rx, their respective sounder circuits, and a centralized nerve center for aggregation (RPyC), timing synchronization (NTP), and coordination (Kafka-Zookeeper). On the NSF POWDER testbed, the nerve center is deployed on a high-availability cluster of four Dell R$740$ compute nodes at the Fort Douglas datacenter, with fault tolerance being a key feature to ensure storage redundancy for the recorded data. As depicted in \cite{ICC}, the Tx is mounted on a building rooftop; while, the Rx is mounted on a van (or a push-cart) that is driven (or pushed) along unplanned routes onsite. Remote monitoring and troubleshooting is provided for validation of geo-positioning, alignment, and power-delay profile samples. The goal of this measurement campaign was to obtain a reasonably large dataset of site-specific measurements for evaluating the propagation characteristics of \SI{28}{\giga\hertz} signals in vehicular communication settings. Thus, our propagation modeling activities included V$2$I measurements under manual, semi-, and fully-autonomous alignment operations traversing nine routes spanning urban and suburban neighborhoods\footnote{Although our platform is capable of double-directional measurements and facilitates easy scalability to MIMO settings, in this campaign, we focus only on \emph{beam-steered measurements} for spatial consistency evaluations.}.

\noindent{\textbf{Post-Processing}}: Using GNURadio utilities, the metadata file corresponding to the route-specific power-delay profile records at the Rx is parsed to extract timestamp information, which is then associated with the geo-positioning and alignment logs at both the Tx and the Rx. The samples in each synchronized power-delay profile segment undergo pre-filtering via a low-pass filter (SciPy FIR implementation), time-windowing, and noise-elimination (via custom peak-search and thresholding heuristics). Coupled with transmission power and antenna gain values, the received power levels obtained from these processed samples allow the visualization of pathloss maps on the Google Maps API (rendered via Bokeh), and the evaluation of pathloss behavior as a function of Tx-Rx distance, with validations against the $3$GPP TR$38.901$ and ITU-R M$.2135$ standards~\cite{MacCartneyModelsOverview}. The SAGE algorithm~\cite{SAGE} is used to extract multipath parameters, which facilitates RMS AoA direction spread studies~\cite{Indoor60G}. Under Tx-Rx distance and Tx-Rx alignment variations, we probe signal decoherence patterns via the spatial/angular autocorrelation coefficient~\cite{MacCartneySpatialStatistics}.
\vspace{-3mm}

% The setup procedures involved in our Wireless InSite ray-tracing simulations
\section{3D Ray-Tracing: Wireless InSite Setup}\label{S3}

% Simulations: Pathloss and Spatial Consistency
\section{Pathloss \& Spatial Consistency}\label{S4}

% Simulations: Multipath Clustering and Channel Modeling
\section{Multipath Clustering \& Channel Modeling}\label{S5}

% Concluding remarks and Future work
\balance
\section{Conclusion}\label{S6}
In this work, via Wireless InSite $3$D ray-tracing simulations, we present physically-interpretable analyses of the pathloss, spatial decoherence, multipath clustering, and Doppler shift evaluations on \SI{28}{\giga\hertz} propagation measurements obtained during our V$2$X propagation modeling campaign on the NSF POWDER testbed. Specifically, employing terrain, structural, and foliage models corresponding to the NSF POWDER site in the ray-tracing evaluations, our investigations involving the pathloss behavior of mmWave signals along OLoS and NLoS links, spatial decorrelation characteristics under distance and alignment effects, intra- and inter-cluster multipath characteristics, and Doppler spectra attributes in vehicular networks, enable us to empirically map such signal propagation observations to their physical interaction objects.
\vspace{-3mm}

% References (main.bib)
\balance
\bibliographystyle{IEEEtran}
\bibliography{IEEEabrv,main} 

\end{document}
% Content ends